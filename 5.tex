\section{Suppose that galactic supernovae obey Poissonian statistics.
The mean number of supernovae per century is 1/3.}

\begin{itemize}
    \item What is the most likely date for the next supernova?

    Poissonian statistics means we have a pdf of the form
    $P(k) = \frac{(R t)^k e^{-R t}}{k!}$,
    where $k$ is the number of supernova observed in time $t$ (measured in centuries),
    and $R = 1/3$ is the mean number of supernovae in a century.

    To find the probability that a supernova happens at a particular time, consider the probability that no supernovae occur in time $T$.

    $$P(k=0;T) = e^{-R T}$$.

    So the probability of having a supernova by time $t$ is given by
    
    $$P(\text{1st supernova within time }T) = 1 - e^{-R T}$$

    That is, the probability that the first supernova occurs before $T$.

    $$P(t_s \leq T) = 1 - e^{-R T}$$

    This is a cumulative probability function, differentiate to get the distribution for $t_s=T$.

    $$P(t_s=T) = R e^{-R T}$$

    This function is monotonically decreasing, so the most likely date is today, Sept 22 2022.

    \item What is the probability distribution for the length of the interval between
    now and the next galactic supernova?

    It's what we had above (an exponential), the probability of the next supernova happening a time $T$ away from now is given by:

    $$P(T) = R e^{-R T}$$

\end{itemize}


