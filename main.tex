\documentclass[11pt,letterpaper]{article}
\usepackage{fullpage}
\usepackage[top=2cm, bottom=4.5cm, left=2.5cm, right=2.5cm]{geometry}
\usepackage{amsmath,amsthm,amsfonts,amssymb,amscd}
\usepackage{lastpage}
\usepackage{fancyhdr}
\usepackage{mathrsfs}
\usepackage{physics}
\usepackage{todonotes} 
\usepackage{subfiles}
\usepackage{hyperref}
\usepackage{enumitem}
\usepackage{url}
\hypersetup{colorlinks,breaklinks,
            urlcolor=[RGB]{0,0.5,0.5},
            linkcolor=[RGB]{0,0.5,0.5}}

\setlength{\parindent}{0.0in}
\setlength{\parskip}{0.05in}


\renewcommand{\d}{\partial}
\renewcommand{\L}{\mathcal{L}}
\newcommand{\B}{\mathcal{B}}
\newcommand{\0}{\mathcal{O}}

\begin{document}

\thispagestyle{fancyplain}
\headheight 35pt
\author{Callum McCracken, 20334298}
\title{PHYS 509C Assignment 1}
\date{September 22, 2022}
\rfoot{\small\thepage}
\headsep 1.5em

\maketitle
\Large{Code for this assignment is here:

\url{https://github.com/callum-mccracken/PHYS-509C-A1}

It's in a bit of a strange format since I make it write the LaTeX file that I use for making the document you're reading, but here are the highlights:

\begin{itemize}
    \item Open the file with \texttt{numpy.loadtxt()}
    \item Get the mean with \texttt{numpy.mean()}
    \item Get the standard deviation with \texttt{numpy.std()}
    \item Get the correlation coefficient with \texttt{numpy.corrcoef()}
    \item Get the skew with \texttt{scipy.stats.skew()}
    \item Use \texttt{scipy.stats.chi2.pdf()} for the chi-squared PDF
    \item Integrate using \texttt{scipy.integrate.quad()}
\end{itemize}
}

\newpage

\subfile{1}
\newpage
\subfile{2}
\newpage
\subfile{3}
\newpage
\subfile{4}
\newpage
\subfile{5}
\newpage
\subfile{6}


\end{document}
